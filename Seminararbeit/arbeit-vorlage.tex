\input{arbeit-vorlage-praeambel.tex} % Importiere die Einstellungen aus der Präambel
% hier beginnt der eigentliche Inhalt
\begin{document}
\pagenumbering{Roman} % Seitenummerierung mit großen römischen Zahlen 
\pagestyle{empty} % kein Kopf- oder Fußzeilen auf den ersten Seiten

% Titelseite
\clearscrheadings\clearscrplain
\begin{center}
\begin{Huge}
Fachhochschule Wedel\\
\vspace{3mm}
\end{Huge}
{\Large Studiengang Medieninformatik}\\

\vspace{20mm}
\begin{Large}
Eine Einführung in die prozedurale Landschaftsgenerierung\\
\end{Large}
\vspace{8mm}
Seminararbeit\\
\vspace{0.4cm}
\vspace{2 cm}
Tjark Smalla \\
Matrikel-Nummer 100554\\
\vspace{8cm}
\begin{tabular}{rl}
{\bfseries Betreuer} & Prof. Bohn\\
\end{tabular}

\end{center}
\clearpage


\pagestyle{useheadings} % normale Kopf- und Fußzeilen für den Rest

\tableofcontents % erstelle hier das Inhaltsverzeichnis
\listoffigures % erstelle hier das Abbildungsverzeichnis
\listoftables % erstelle hier das Tabellenverzeichnis

% richtiger Inhalt
\chapter{Einleitung}
\pagenumbering{arabic} % ab jetzt die normale arabische Nummerierung

Während die künstliche Erzeugung von Landschaften\footnote{Landschafts-Synthese} in der Computergrafik schon seit den 80gern ein Thema ist, beschäftigte sich auch Hersteller von Computerspielen immer wieder mit dieser Technik. Insbesondere in den letzten 10 Jahren begannen immer mehr Entwickler diese Technik anzuwenden.

Die prozedurale Generierung ihrer Landschaften bietet ihnen viele Vorteile. 
Durch die mittlerweile sehr großen Spielwelten ist es nicht unüblich bis zu 300 Leute über 2 Jahre an nur einem großen Projekt zu beschäftigen. Ein großer Teil dieser Zeit wird in die Ausgestaltung der Spielwelt gesteckt. Prozedurale Techniken erlauben es ihnen realistische Spielwelten zu erschaffen die nahezu unendlich groß sind.
Als bekanntestes Beispiel ist hier sicherlich das Spiel Minecraft vom Studio Mojang zu nennen, welches einen Großteil seiner Faszination aus der komplett prozedural erzeugten veränderbaren Landschaft zieht.

Im folgenden werden 3 bewährte Algorithmen zur Erzeugung von Landschaften vorgestellt und daraufhin untersucht, in wie weit sie zur Speicherung einer nahezu unendlich großen Landschaft wie in Minecraft geeignet sind.

Zuerst wird der \emph{Diamond-Square} Algorithmus\cite{DiamondSquare} vorgestellt, welchen man verallgemeinert auch als einen Polygon unterteilungs Algorithmus bezeichnen kann. Danach wird kurz die \emph{Spektalsynthese} mit der Fourier-Methode erläutert bevor es eine Einführung in die Welt der Rauschfunktionen gibt. 
Neben der Synthese von Landschaften sind diese Algorithmen vielseitig einsetzbar. Insbesondere bei der bereits erwähnten Kategorie der Rauschfunktionen ist davon auszugehen, dass sie auch in proprietärer 3D-Software wie Computerspielen trotz ihres Alters noch als wichtiger Algorithmus genutzt werden. Dies liegt vor allem an ihrer Flexibilität sowie Skalierbarkeit in mehreren Dimensionen durch die auch Wolken, Feuer, Rauchverwirbelung und sogar Fell auf einer Oberfläche erfolgreich dargestellt wurden \cite{texturingAndModeling}\footnote{Wolken: Kapitel 9, Feuer(S,125) \& Rauchverwirbelung: Kapitel 12, Rauch, Wolken, Fell: S.297-302}.
Neben der Bewertung soll dabei insbesondere die Erklärung der Grundlagen dieser Algorithmen Zielsetzung sein. Zu diesem Zwecke wird neben der Erklärung im Unterkapitel \emph{"Der Algorithmus"} jedes Verfahren auf seine Erweiter- bzw. Anpassbarkeit(\emph{"Flexibilität"}) hin untersucht, bevor es eine abschließende \emph{"Bewertung im Rahmen der Fragestellung"}. In diesem letzten Schritt wird das Verfahren daraufhin untersucht, in wie weit es effektiv zur Speicherung und Auswertung einer nahezu unendlich großen Landschaft geeignet ist.

Der Diamond-Square Algorithmus sowie verschiedene Rauschfunktionen sind in dem dieser Arbeit beiliegendem Unity-Projekt in C\# bzw. HLSL\footnote{High Level Shading Language} implementiert und lassen sich durch Unity auf Windows, Unix und Mac OS basierten Betriebssystemen kompilieren.

\section{Datenstrukturen zur Landschaftsspeicherung: Höhenfelder vs. Voxel}
Landschaften werden in der Regel entweder in einem Höhenfeld oder in Voxeln gespeichert.
Der wesentliche Unterschied der beiden Techniken liegt in ihren Dimensionen. Das Höhenfeld speichert in einem ganzzahligen, zweidimensionalen Koordinatensystem zu jedem Punkt den dazugehörigen Höhenwert der Landschaft, während ein Voxel den Dichtewert eines Punktes in einem dreidimensionalem Koordinatensystem darstellt. Beide Datenstrukturen haben den Vorteil, dass sich die Position jedes Wertes in der Landschaft implizit aus der Position zu seinen Nachbarn bestimmen lässt. Eine zusätzliche Speicherung seiner Koordinate ist also nicht notwendig. Anders wäre es bei Polygonen, wo jeder Vertex eine Koordinate im Raum hat. Sie sind daher gut für Szenen mit viel leerraum geeignet, welcher allerdings in den hier vorgestellten Datenstrukturen zu unnötig gespeicherten Daten führe würde

\subsection{Höhenfelder}
Die meisten Algorithmen beziehen sich auf die Erzeugung von Landschaften mit Hilfe von Höhenfeldern. Dies ist vermutlich vor allem dem geschuldet, dass viele Algorithmen recht alt sind und sich Voxel basierte Verfahren aufgrund des Speicherplatzmangels nicht lohnten.
Auch wenn Höhenfelder bereits zu sehr realistischen Ergebnissen großer Landschaftsstriche führen, haben sie einen entscheidenden Nachteil. Da pro Punkt im Koordinatensystem nur ein Höhenwert gespeichert werden kann sind Höhlen oder Überhänge nicht möglich.

Aufgrund der vereinfachten Algorithmen und der großen Verbreitung wird sich im folgendem auf Höhenfelder beschränkt. Eine Erweiterung in die dritte Dimension ist mit dem Diamond-Square Algorithmus oder den Rauschfunktionen jedoch problemlos möglich.
Ein weiterer Vorteil von Höhenfeldern ist, dass sich durch die simple Verbindung nebeneinanderliegender Punkte sehr einfach ein Polygon erzeugen lässt, welches Hardwarebeschleunigt gerendert werden kann.

\begin{figure}
	\centering
	\includegraphics[width=\textwidth]{images/heightfield_rendered.png}
	\caption{Gegenüberstellung einer Heightfield als Textur(links) und der resultierenden Landschaft. Bildquelle: http://tinyurl.com/zu7gond}\label{img.heightfield}
\end{figure}

\subsection{Voxel}
Der Begriff Voxel leitet sich aus den Begriffen Volumen und Pixel ab.
Durch die Speicherung in einem dreidimensionalem Koordinatensystem wird nun auch die, bei den Höhenfeldern explizit gespeicherte, Höhe eines Punktes implizit gespeichert. Dies ermöglicht es eine weitere Information für jeden Punkt explizit zu speichern. In der Regel ist dies ein Dichtewert der es ermöglicht verschiedene Arten von Materialien oder die Opazität zu simulieren.

Um Voxel hardwarebeschleunigt zu rendern müssen auch diese in ein Polygon umgewandelt werden. Dies funktioniert normalerweise über Raycasting oder den Marching Cube Algorithmus.

\begin{figure}
	\centering
	\includegraphics[width=\textwidth]{images/voxel_rendered.jpg}
	\caption{Gerenderte Voxellandschaft mit Überhängen erzeugt aus mehreren Rauschfunktionen. Bildquelle: http://tinyurl.com/jq8vta8}\label{img.heightfield}
\end{figure}

\section{Implizite vs. explizite Funktionen}
Alle hier vorgestellten Methoden lassen sich in 2 Gruppen einteilen: implizite und explizite Funktionen.
Während eine explizite Funktion alle Höhenpunkte auf einmal berechnet lässt sich die implizite Funktion für jeden Punkt, also jede Koordinate, isoliert auswerten. 

Durch die Unabhängigkeit der Berechnung für jeden einzelnen Punkt lassen sich implizite Algorithmen sehr effizient parralel auf einer GPU berechnen\footnote{Siehe Beispielimplementierung in einem Vertex-Shader}. Dies ermöglicht die Ausführung zur Laufzeit, während explizite Methoden in der Regel vor oder bei Programmstart einmalig berechnet werden und deren Ergebnisse in einer Textur gespeichert werden.
Dies hat den Vorteil, dass der Speicherbedarf enorm sinkt.

Ein Einsatzgebiet für diese Technik ist das Bump-Mapping bzw. Displacement Mapping bei der zusätzliche Höhenwerte auf ein Objekt durch Shading oder neue Vertices auf der Objektoberfläche hinzugefügt werden\cite{displacementNStuff}. Da moderne Echtzeitspiele immer mehr und immer größere Texturen verwenden steigt der Bedarf an Speicher enorm wenn für jede Textur noch eine Normal/Bump/Displacement Map gespeichert werden muss. Implizite Methoden erlauben es, anstatt der Texturen einige Parameter in Form von Floats und Integern zu speichern.
Auch eine Anpassung des Detailgrades ist zur Laufzeit ohne Probleme möglich, während bei der Detailgrad bei expliziten Methoden von der Auflösung der Textur abhängt\footnote{Diese Eigenschaften lassen sich zwar auch durch die Berechnung von expliziten Methoden zur Laufzeit erreichen, jedoch lassen diese sich wie erwähnt nicht effektiv durch die GPU beschleunigen wodurch die Berechnung innerhalb eines Frames unperformant ist.}.

Der Diamond-Square Algorithmus sowie die Spektralsynthese gehören zu der Gruppe der expliziten Algorithmen, während die Rauschfunktionen implizit auswertbar sind.

\chapter{Diamond-Square Algorithmus}
Die Vorteile des Diamond-Square Algorithmus liegen insbesondere in seinen, trotz seiner einfachen Implementierung, optisch plausiblen Ergebnissen. 
Der Algorithmus unterteilt sich pro Rekursionschritt in 2 Phasen\footnote{Auch \emph{Steps} genannt}: der Diamond und der Square Step. Das zu berechnende Höhenfeld muss dabei in der Auflösung eine Breite bzw. Höhe von $2n+1$ wobei $ n\in\mathbb{N}$ aufweisen.

\subsection{Der Algorithmus}
Zur Initialisierung werden den vier Eckpunkte des zu berechnenden Höhenfeldes jeweils zufällige Höhenwerte zugewiesen. Außerdem wird eine von der Rekursionstiefe abhängige Offset-Funktion $O(k)=(rnd()*2-1)/2^k$ definiert, wobei rnd() einen Zufallswert zwischen 0-1(inklusive) zurückgibt.

Im Diamond Step wird nun der Mittelpunkt des Höhenfeld gesucht und mittels einer beliebigen Interpolation\footnote{In der Beispielimplementierung wurde eine billineare Interpolation gewählt} zwischen den 4 Höhenwerten der Eckpunkte ein Wert gefunden, welcher dann mit der Offset-Funktion zufällig verschoben wird.
\begin{figure}
	\centering
	\includegraphics[width=\textwidth]{images/Diamond_Square.png}
	\caption{2 Rekursionschritte des Diamond Square Algorithmus bei einem quadratischen Höhenfeld mit der Auflösung 5x5}\label{img.dmsquare}
\end{figure}
Bei dem Square Step werden nun die jeweiligen mittleren Randpunkte zwischen den bereits initialisierten Eckpunkten durch eine einfache Interpolation zwischen den 2 nächstliegendsten Eckpunkten berechnet.
Wie in \autoref{img.dmsquare} zu sehen ergeben sich daraus 4 weitere Vierecke mit berechneten Eckpunkten, auf die der Algorithmus wieder angewandt werden kann.

\subsection{Flexibilität}
Das resultierende Höhenfeld lässt sich einfach durch eine Veränderung der Offset-Funktion relativ flexibel anpassen. So besteht zum Beispiel die Möglichkeit, die Offset-Funktion von dem interpolierten Höhenwert des aktuellen Punktes abhängig so machen und dadurch ein heterogenes Landschaftsbild zu erzeugen.

\subsection{Bewertung im Rahmen der Fragestellung}
Zur Speicherung des resultierenden Höhenfeldes reicht es aus den Seed\footnote{Eingangswert für einen Pseudozufallsgenerator. https://www.wikiwand.com/en/Random\_seed} des Pseudozufallsgenerators sowie die vier initialen Werte zu speichern. Eine Landschaft wie sie etwa Minecraft bietet wäre mit diesem Algorithmus aus 2 Gründen allerdings nicht möglich. Wie bereits besprochen verwendet Minecraft Voxel, welche mit dem Diamond Square Algorithmus allerdings nicht berechnet werden können. Außerdem würde die Generierung der Spielwelt besonders für Spieler mit Leistungsschwachen Computern sehr lange dauern. Um das zu vermeiden könnte man die Landschaft in rechteckige Patches\footnote{Kleine Abschnitte der Landschaft einzeln berechnen und aneinanderfügen} aufteilen, die erst berechnet werden wenn der Spieler auf sie tritt. Da der Pseudozufallsgenerator jedoch seine Werte immer in der gleichen Reihenfolge ausgibt wäre die Anordnung der Patches zueinander abhängig von der Reihenfolge in der sie geladen werden\label{Patches}.

\lstinputlisting[language=csh, title=Diamond-Square Implementierung C\#]{data/diamondSquare.cs}

\chapter{Fourier-Spektralsynthese}
Bei der Spektralsynthese wird das Höhenfeld in der Frequenz-Domäne modelliert und durch die inverse Fourier Transformation in ein Höhenfeld in der Raum-Domäne umgewandelt. 

Als ersten Schritt sollten sich Gedanken darüber gemacht werden, welche Frequenzanteile für einen plausiblen Eindruck notwendig sind und wie stark diese ausgeprägt sein sollten. Eine Landschaft etwa hat Niederfrequente Anteile mit großer Amplitude die Hügel und Täler formen und Höherfrequente Anteile mit niedriger Amplitude durch die unregelmäßigkeiten und die Rauheit des Bodens simuliert wird.

Um das Höhenfeld zu modellieren wird weißes Rauschen erzeugt welches im Anschluss durch einen beliebigen Filter die ungewünschten Frequenzanteile entfernt bzw. deren Amplitude verringert.
%TODO QUellen hinzufügen
Weißes Rauschen bezeichnet ein Signal welches aus im der Frequenz-Domäne allen Frequenzen eine konstante Amplitude zuordnet. Für den Zweck der Höhenfeldsynthese reicht uns eine Annäherung vollkommen aus. Dazu wird in der Raum-Domäne ein Bild erzeugt in dem der Farbwert für einen Pixel durch einen einfachen, C-ähnlichen Pseudozufallsgenerator erzeugt wird. Dieses verhält sich, in die Frequenz-Domäne übertragen, ähnlich wie weißes Rauschen.

Der Filter den wir nun auf dieses Rauschen anwenden bestimmt entscheidend das Aussehen des späteren Höhenfeldes. Plausible Ergebnisse lassen sich mit dem Filter $f=1/f^\mathbb{b}$%TODO beta
erzeugen. Das resultierende Frequenz Bild entspricht bei $beta=2$ dem von Fractale-Brownian Motion\footnote{Auch Pinkes-Rauschen}.
In %TODO Bilder
sieht man das in 3D visualisierte Ergebnis des resultierenden Höhenfeldes.

Die Spektralsynthese ist sehr flexibel anpassbar, da sich mit der Änderung des Filters beliebig viele Effekte mit wenig Implementierungsaufwand erzeugen lassen.

%TODO Bilder 

\chapter{Noise}
Synthetisch erzeugtes Rauschen \emph{(engl. Noise)} erweist sich als hilfreiches Mittel zur Erzeugung von zufällig erscheinenden Strukturen.
Als wohl bekannteste Implementierung ist hier die Implementierung von Ken Perlin\cite{PERLIN1985} zur Erzeugung einer Marmortextur auf einer Vase zu nennen\footnote{Auch als \emph{Perlin-Noise} bezeichnete Implementierung von Gradient Noise in 3-D}.

Neben umfangreichen Anpassungsmöglichkeiten durch verschiedene Parameter ist die Performance dieses Verfahrens ein entscheidender Grund für die Nutzung. Noise verbraucht extrem wenig Speicher, ist relativ einfach zu berechnen und ist zu jeder Zeit an einer beliebigen Stelle auswertbar, was es auch für Echtzeitanwendungen geeignet macht.\cite{H.Hauser2010}

Dieses Kapitel soll ein grundlegendes Verständnis über Noise-Funktionen bieten. Dazu werden zuerst grundlegende Komponenten, welche jeder Implementierung zugrunde liegen, erläutert. Anschließend werden \emph{Value-}\ref{Value-Noise}, \emph{Gradient-Noise}\ref{Gradient-Noise} sowie \emph{Fractal-Noise}\ref{Fractal-Noise} erklärt, bevor es einen Ausblick auf verschieden Abwandlungen von \emph{Fractal Noise} gibt.

Weitere ausführliche Beschreibung in \cite{BurgerGradientNoise2008} (Gradient-Noise) und in \cite{simplexNoise} zu \emph{Simplex-Noise} welches eine, besonders in höheren Dimensionen, performantere Implementierung von Perlin Noise darstellt.

\section{Grundlagen}
\subsection{Lattice-Function}\label{latticeFunc}
Der erste Schritt zur Erzeugung von Noise ist in der Regel eine sogenannte \emph{Lattice(Gitter)-Funktion}\cite{fractalsAndChaos} der Form \begin{math}l(\vec{k}): {Z}^n \mapsto [-1 - 1]\end{math}\label{latticeFunc}.
Diese dient zur Beschreibung eines Gitters, welches die Form unserer zukünftigen Noise-Funktion bestimmen wird. Die Funktion muss dabei unbedingt deterministisch sein\footnote{Siehe\ref{Fractal-Noise}, sie muss also für jede Koordinate eines Gitterpunktes immer denselben Funktionswert liefern.}. Perlin verwendet bei seiner Implementierung eine Hashfunktion die auf ein, mit zufälligen Werten gefülltes, Array zugreift. Ein Zugriff auf dieses Array mittels simpler Modulo Operation hätte zur Folge, dass sich im Höhenfeld wiederkehrende Strukturen zeigen würden, sobald die Funktion über den Rahmen des Arrays hinausgreift.


\subsection{Interpolation und Fade-Function}
Um aufbauend auf der Lattice-Funktion\ref{latticeFunc} eine Funktion $S(\vec{x}): \mathbb{R}^n\mapsto\mathbb{R}, \vec{x}\in \mathbb{Z}^n$\label{S} zu definieren wird zwischen benachbarten Gitterpunkten lokal interpoliert. Dafür wird eine sogenannte Fade-Function\cite{fadeFunction} der Form $f(t): \mathbb{R}\mapsto\mathbb{R}$ mit $t\in[0, 1]$ definiert, welche den Übergang zwischen den Gitterpunkten steuert.

Um überhaupt eine stetige Noise-Funktion zu ermöglichen, muss 
\begin{equation}
f(0) = 0 \land f(1) = 1
\end{equation} gelten.
Damit der Übergang zwischen den Gitterpunkten möglichst glatt und damit natürlich wirkt, sollte jedoch eine Stetigkeit von $C^2$ an den Übergängen und damit die Eigenschaften 
\begin{equation}
	f'(0) = f'(1) = 0 = f''(0) = f''(1)
\end{equation} 
gelten.

Dafür wird im folgenden das Polynom $f(t) = 6t^5-15t^4+10t^3$ benutzt, welches auch in Perlins Referenzimplementierung Verwendung findet\cite{BurgerGradientNoise2008} und alle Eigenschaften erfüllt.
\begin{figure}[!hbtp]%
	\centering
	\begin{tikzpicture}
		\begin{axis}
			[ 
				xlabel=$t$,
				ylabel={$f(t) = 6t^5-15t^4+10t^3$},
				ymin=0,
				ymax=1,
				xmin=0,
				xmax=1,
				restrict x to domain=0:1
			] 
			
			\addplot[no markers, blue, smooth, domain = 0:1] {x * x * x * (x * (x * 6 - 15) + 10)}; 
		\end{axis}
	\end{tikzpicture}
	\caption{Fade-Function}
\end{figure}

Die mit $f(t)$ gebildete, interpolierende Funktion $S(\vec{x}) = noise(\vec{x})$ definiert nun - im einfachsten Fall - die Noise-Funktion.


\section{Value-Noise}\label{Value-Noise}
Value-Noise ist die wohl naivste Implementierung einer Noise-Funktion. Bei ihr werden die Gitterpunktwerte, welche durch die \emph{Lattice-Funktion}\ref{latticeFunc} erzeugt wurden, als Höhenwerte interpretiert.

Untenstehend ist eine Implementierung in C\# für eine 2-Dimensionale Rauschfunktion zu sehen.
Noise lässt sich problemlos in mehrere Dimensionen skalieren. Einzig die Interpolation der Werte muss hier angepasst werden. In der Implementierung ist zu sehen, wie die Gitterpunktwerte - ähnlich einer \emph{billinearen Interpolation} - mit der \emph{Fade-Function} interpoliert werden.

\lstinputlisting[language=csh, title=Value-Noise Implementierung C\#]{data/valueNoise.cs}\label{valueNoise.cs}

\section{Gradient-Noise}\label{Gradient-Noise}
Der vorher erwähnte Value-Noise kann, je nach Parameterwahl, noch ein unruhiges Rauschen erzeugen. Um die Übergänge zwischen den Gitterpunktwerten noch sanfter und damit natürlicher aussehen zu lassen wurde der Gradient-Noise erfunden.
Hier werden die Gitterpunktwerte nicht als Höhenwerte, Gradienten an den Nullstellen der Noise-Funktion gesehen.
Es ergibt sich also:
\begin{equation}
noise(\vec{k}) = 0 \land 
noise'(\vec{k}) = S(\vec{k}),  k \in \mathbb{Z}^n  .
\end{equation}

In der untenstehenden 2D C\# Implementierung ist zu sehen, wie zuerst ein Höhenwert für den aktuellen Punkt  $\vec{x}$ über das Skalarprodukt\footnote{engl. Dot-Product} zwischen dem Gradienten und der relativen Position $\begin{pmatrix}tx\\ty\end{pmatrix}  tx, ty\in [0-1]$ und anschließend über die bekannte Interpolation berechnet wird.

\lstinputlisting[language=csh, title=Gradient-Noise Implementierung C\#]{data/gradientNoise.cs}
 

\section{Fractal-Noise}\label{Fractal-Noise}
%TODO Wieso Fractal? Was ist ein Fraktal?
Die bisher behandelten Noise-Funktionen erzeugen zwar natürlich erscheinende Zufallswerte, wenn man diese jedoch auf eine Heightmap überträgt wird deutlich, dass es ihr an Details fehlt um natürlich zu wirken.

Um den Detailgrad der Noise-Funktion beliebig zu erhöhen, wird die Noise-Funktion mit einer gestauchten, in der Amplitude verringerten, Version ihrer selbst addiert. 
Dieses Verfahren lässt sich beliebig oft anwenden, was einen beliebig hohen Detailgrad erlaubt.

In \cite{Saupe} wird daher folgende Formel definiert:
\begin{equation} \label{eq.fractalnoise}
	\mathbb{H}(\vec{x}) = \sum_{k=k_0}^{k_1}\frac{1}{r^{kH}}S(r^k\vec{x}).
\end{equation}

Wobei $H=2-D$ der Hurst-Exponent ist\cite{tuMuenchen} und $D=-\frac{log(\frac{1}{k1})}{log(r)}$\cite{fraktDim} die fraktale Dimension.

Durch die Anpassung dieser Parameter lässt sich das Verhalten der Noise-Funktion gezielt steuern. Eine Implementierung in C\# findet sich unten.

\lstinputlisting[language=csh, title=Fractal-Noise Implementierung C\#]{data/fractalNoise.cs}

\section{Noise-Variationen}\label{NoiseVariationen}
Die in \ref{eq.fractalnoise} beschriebene Funktion lässt sich beliebig erweitern und anpassen um bestimmte Effekte zu erzielen. Im folgenden sollen daher einige bekannte und für Landschaften passende Erweiterungen erläutert werden. Diese Auflistung ist nur als eine Auswahl zu verstehen. Es gibt noch zahlreiche weitere Implementierungen wie z.B. Convolution oder auch sparse Convolution-Noise\cite{texturingAndModeling} %TODO wenigstens Kapitel
welche allerdings keinen 

\subsection{Turbulence-Noise}

\subsection{Ridged-Noise}

\subsection{Multifractal}


%TODO Auflistung und Bilder der Ergebnisse









\bibliographystyle{alphadin_martin}
\bibliography{bibliographie}


\chapter*{Erklärung}

Hiermit versichere ich, dass ich die vorliegende Arbeit selbstständig verfasst und keine anderen als die angegebenen Quellen und Hilfsmittel benutzt habe, dass alle Stellen der Arbeit, die wörtlich oder sinngemäß aus anderen Quellen übernommen wurden, als solche kenntlich gemacht und dass die Arbeit in gleicher oder ähnlicher Form noch keiner Prüfungsbehörde vorgelegt wurde.

\vspace{3cm}
Ort, Datum \hspace{5cm} Unterschrift\\

\end{document}