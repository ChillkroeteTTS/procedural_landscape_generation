\chapter{Fazit}
Wenn es um die Erzeugung von simplen, realistisch wirkenden Höhenfeldern vor Programmstart geht ist der Diamon-Square Algorithmus die einfachste Lösung. Die Spektralsynthese ist zwar aufgrund ihrer Nutzung von beliebigen Filtern sehr flexibel, durch den Einsatz der Fourier Transformation gibt es aber keine Möglichkeit der Erzeugung von heterogenem Terrain. Die Noise bzw. Rauschfunktionen zeigen ihren Nutzen besonders ihrer einfachen Implementierung in höheren Dimension und ihrer daraus resultierenden vielseitigen Einsatzgebieten.

Für die Erzeugung von nahezu unendlich großen Landschaften eignen sich insbesondere die Rauschfunktionen. Sie verbrauchen extrem wenig Speicher in der Speicherung der Landschaft und erlauben eine problemlose, parallele, punktuelle Auswertung. 