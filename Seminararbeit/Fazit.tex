\chapter{Fazit}
Wenn es um die Erzeugung von realistisch wirkenden Höhenfeldern vor Programmstart geht ist der Diamon-Square Algorithmus die einfachste Lösung. Die Spektralsynthese ist zwar aufgrund ihrer Nutzung von beliebigen Filtern sehr flexibel, durch den Einsatz der Fourier Transformation gibt es allerdings keine Möglichkeit zur Erzeugung von heterogenem Terrain. Die Noise bzw. Rauschfunktionen zeigen ihren Nutzen besonders in ihrer einfachen Implementierung in höheren Dimension und ihrem daraus resultierenden vielseitigen Einsatzgebieten.

Für die Erzeugung von nahezu unendlich großen Landschaften eignen sich insbesondere die Rauschfunktionen. Neben dem geringen Speicherverbrauch, lässt sich die Funktion nur an der Stelle auswerten, die auch gerendert wird. Während bei den anderen Methoden die Auflösung zum Zeitpunkt der Höhenfeldsynthese feststehen muss ist es mit Rauschfunktionen möglich, die Auflösung zum Renderzeitpunkt zu bestimmen. Dadurch ist eine Veränderung des LOD\footnote{Level of detail - Beschreibt die Auflösung eines 3D-Modells} der Landschaft, abhängig von der Kameraposition, möglich.